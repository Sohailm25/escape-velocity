\documentclass[11pt]{article}

% === Packages ===
\usepackage[margin=1in]{geometry}
\usepackage{amsmath,amssymb}
\usepackage{graphicx}
\usepackage{booktabs}
\usepackage{natbib}
\usepackage[colorlinks=true,linkcolor=blue,citecolor=blue,urlcolor=blue]{hyperref}
\usepackage{xcolor}

% === Title ===
\title{Condition-Dependent Collapse Dynamics in Multi-Turn LLM Self-Play}
\author{Sohail Mohammad}
\date{\today}

\begin{document}
\maketitle

\begin{abstract}
We present a systematic empirical study of conversational collapse dynamics in multi-turn self-play between 7B-parameter language models. Using a preregistered confirmatory design (N=720 trajectories: 4 conditions $\times$ 36 seeds $\times$ 5 repeats, 40 turns each), we characterize collapse behavior across homogeneous self-play configurations (Llama-3.1-8B, Qwen2.5-7B, Mistral-7B-v0.3) and a heterogeneous rotation condition. Baseline execution achieved complete tuple closure (720/720) with full protocol integrity. Collapse dynamics were condition-dependent: Qwen2.5-7B homogeneous self-play exhibited the highest mean collapse rate (0.773), while Mistral-7B showed the lowest (0.141). Heterogeneous model rotation produced intermediate collapse (0.250). The preregistered detector reliability criterion (Cohen's $\kappa \geq 0.80$) was not met ($\kappa = 0.566$; raw agreement 92.8\%). Consequently, all collapse-pattern findings are reported as descriptive rather than detector-validated.
\end{abstract}

% === Introduction ===
\section{Introduction}
Long-horizon language-model interaction can drift into repetitive output regimes that degrade
novelty and responsiveness. Related concerns about repetition, degeneration, and reliability in
generative systems are well documented in single-agent and dialogue settings
\citep{holtzman2019curious,li2016diversity,welleck2020neural}. However, less is known about how
such behavior distributes across controlled multi-turn self-play conditions when protocol settings
are held fixed.

Paper A was designed as a preregistered baseline characterization study rather than an
intervention paper. The confirmatory objective was to estimate condition-level collapse patterns
under a locked detector and to evaluate detector reliability against an independent audit gate.
Because the preregistered reliability threshold was not met ($\kappa=0.566$, threshold 0.80), this
manuscript follows Path B claim governance: transparent downgrade of inferential scope.

Accordingly, our claims are limited to: (i) execution closure and protocol integrity, and
(ii) descriptive, condition-comparative patterns under the locked detector definition. We do
not claim detector validation, causal condition effects, or mitigation efficacy.

% === Methods ===
\section{Methods}

\subsection{Confirmatory design}
The locked confirmatory design was $4 \times 36 \times 5$: 4 conditions (HOMO\_A, HOMO\_B,
HOMO\_C, HETERO\_ROT) $\times$ 36 seed prompts $\times$ 5 repeats, yielding 720 unique
trajectories. Each trajectory ran for exactly 40 turns with no early stopping. The trajectory
unit is a tuple $(\text{condition} \times \text{seed} \times \text{repeat})$.

\subsection{Conditions and model mapping}
\begin{itemize}
    \item \textbf{HOMO\_A:} Llama-3.1-8B-Instruct (rev \texttt{0e9e39f2})
    \item \textbf{HOMO\_B:} Qwen2.5-7B-Instruct (rev \texttt{a09a3545})
    \item \textbf{HOMO\_C:} Mistral-7B-Instruct-v0.3 (rev \texttt{c170c708})
    \item \textbf{HETERO\_ROT:} round-robin rotation across all three model families
\end{itemize}
Generation settings were fixed at temperature 0.7, top-$p$ 0.95, and max tokens 256.

\subsection{Collapse detector (locked operating point)}
Collapse was detected with an embedding-based rule set using
\texttt{sentence-transformers/all-MiniLM-L6-v2} \citep{reimers2019sentencebert}. At turn $t$,
periodicity features were defined as $s_{1,t}=\cos(e_t,e_{t-1})$ and
$s_{2,t}=\cos(e_t,e_{t-2})$, with thresholds $s_{1,t}\geq0.92$ or $s_{2,t}\geq0.90$ sustained
for at least three consecutive turns. Drift constraints were
$d_1\leq0.08$, $d_2\leq0.10$, where $d_k=1-\cos(\cdot)$. A turn was labeled collapsed only when
periodicity and low-drift criteria co-occurred.

\subsection{Reliability gate protocol}
The preregistered reliability gate required Cohen's $\kappa \geq 0.80$ on a stratified
180-window audit sample. Two independent LLM raters labeled windows under a locked rubric.
Final gate result was \textbf{NOT MET}: $\kappa=0.566$ (167/180 agreement; 92.8\% raw agreement).

\subsection{Artifacts and reproducibility contract}
The canonical source artifact for confirmatory summaries is the frozen baseline matrix
(coverage matrix hash recorded in the submission reproducibility index and freeze note).
Tables and figures in this manuscript were generated from that frozen artifact.

% === Results ===
\section{Results}

\subsection{Baseline completion}
The confirmatory baseline matrix completed in full: 720/720 tuples succeeded with protocol
integrity (all 40 turns, no early stopping).

\subsection{Condition-level collapse patterns}

\begin{table}[h]
\centering
\caption{Condition-level collapse summary (N=180 each).}
\label{tab:condition_summary}
\begin{tabular}{lcccc}
\toprule
Condition & Mean & Median & SD & N \\
\midrule
HOMO\_A (Llama)    & 0.457 & 0.475 & 0.311 & 180 \\
HOMO\_B (Qwen)     & 0.773 & 0.850 & 0.210 & 180 \\
HOMO\_C (Mistral)  & 0.141 & 0.025 & 0.208 & 180 \\
HETERO\_ROT        & 0.250 & 0.138 & 0.281 & 180 \\
\bottomrule
\end{tabular}
\end{table}

\begin{figure}[h]
\centering
\includegraphics[width=0.9\textwidth]{figures/figure_1_mean_collapse_by_condition.pdf}
\caption{Mean collapse rate by condition with standard deviation error bars.}
\label{fig:mean_collapse}
\end{figure}

\begin{figure}[h]
\centering
\includegraphics[width=0.9\textwidth]{figures/figure_2_collapse_distribution.pdf}
\caption{Collapse rate distribution by condition (boxplot with jittered strip overlay).}
\label{fig:collapse_dist}
\end{figure}

\subsection{Detector reliability}
The preregistered reliability criterion ($\kappa \geq 0.80$) was not met. Although raw agreement
was high (92.8\%), estimated reliability remained at $\kappa=0.566$ under high collapse
prevalence in the audit set.

% === Discussion ===
\section{Discussion}
Under fixed protocol settings, collapse dynamics varied substantially by condition. The observed
ordering (HOMO\_B $>$ HOMO\_A $>$ HETERO\_ROT $>$ HOMO\_C) was stable in this dataset and provides a
useful descriptive baseline for follow-on work.

Path B constraints are central to interpretation. Because the preregistered reliability gate was
not met, these results should be used as detector-defined descriptive evidence only. They support
prioritizing stronger human-labeled reliability procedures and alternative detector formulations
before any confirmatory or causal claim expansion.

% === Limitations ===
\section{Limitations}
The preregistered detector reliability gate was not met ($\kappa=0.566$; threshold 0.80).
Three rubric iterations were conducted (v1.0: $\kappa=0.011$; v2.1: $\kappa=0.519$;
v3.0: $\kappa=0.566$). The gap between raw agreement (92.8\%) and $\kappa$ is consistent with
strong prevalence skew. A prevalence-adjusted statistic (PABAK = 0.856) is reported as
supplementary sensitivity information, not as gate replacement.

Consequently, all findings are descriptive and condition-comparative rather than
detector-validated.

% === Conclusion ===
\section{Conclusion}
We provide a fully closed confirmatory baseline (720/720 tuples) for condition-dependent
collapse behavior in 7B-model self-play under a locked protocol. The present contribution is a
reproducible descriptive benchmark under Path B claim governance; detector-validation and
intervention efficacy remain open tasks for future work.

\bibliographystyle{plainnat}
\bibliography{references}

\end{document}
